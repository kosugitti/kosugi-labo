\documentclass{ltjsarticle}
\usepackage{luatexja-fontspec}
\usepackage{amsmath} % 数式用
\usepackage{bm} % ベクトル・行列用
\usepackage{graphicx} % 図の挿入用
\usepackage{caption} % キャプションのカスタマイズ用
\captionsetup{justification=raggedright,singlelinecheck=false} % キャプション左揃え
\usepackage[top=25truemm,bottom=25truemm,includefoot,includehead,heightrounded]{geometry} % 余白設定

\usepackage{hyperref} % ハイパーリンク用
% Hyper Linkを目立たせない呪文
\hypersetup{
setpagesize=false,
 bookmarksnumbered=true,%
 bookmarksopen=true,%
 colorlinks=true,%
 linkcolor=blue,
 citecolor=pinegreen,
 urlcolor=magenta
}


\usepackage[backend=biber,style=jpa]{biblatex}

\addbibresource{sample.bib}

% タイトル、著者、日付
\title{LaTeX入門}
\author{小杉考司}
\date{\today}

\begin{document}
% タイトル出力
\maketitle
\tableofcontents % 目次出力

\newpage

\section{LaTeX文書の基本構造}

LaTeX文書は大きく「プリアンブル」と「本文」に分かれる。

\subsection{プリアンブル}
\verb|\documentclass{...}|から\verb|\begin{document}|までの部分をプリアンブルと呼ぶ。
文書全体の設定を行う場所である。

\subsubsection{ドキュメントクラス}
\begin{verbatim}
\documentclass{ltjsarticle}
\end{verbatim}
文書の種類を指定する。LuaLaTeXで日本語を使う場合は以下のいずれかを使う:
\begin{description}
    \item[ltjsarticle] 論文・レポート向け(セクション番号あり)
    \item[ltjsreport] 長めのレポート向け(章から始まる)
    \item[ltjsbook] 書籍向け
\end{description}

\subsubsection{パッケージの読み込み}
\verb|\usepackage{パッケージ名}|で機能を追加する。このテンプレートで使用しているパッケージは次のとおりである。
\begin{description}
    \item[luatexja-fontspec] 日本語フォントの設定
    \item[amsmath] 数式環境の拡張
    \item[bm] 太字の数式記号(ベクトル・行列用)
    \item[graphicx] 図の挿入
    \item[caption] キャプションのカスタマイズ
    \item[geometry] 余白の設定
    \item[hyperref] ハイパーリンク・PDF目次
    \item[biblatex] 引用文献管理
\end{description}

\subsubsection{タイトル情報}
\begin{verbatim}
\title{論文タイトル}
\author{著者名}
\date{\today}  % または \date{2025年1月7日}
\end{verbatim}
これらはプリアンブルで設定し,本文中で\verb|\maketitle|を呼ぶと出力される。
\verb|\today|は自動的に今日の日付になる。

\subsection{本文}
\verb|\begin{document}|から\verb|\end{document}|までが本文である。
本文はプレーンテキストなので,そのまま書けば良い。ちなみに,よくある質問として「改行できない」「字下げできない」というのがあるが,改行は空行を入れればよく,字下げは自動的に行われるので特に意識する必要はない。

\subsubsection{タイトルの出力}
\begin{verbatim}
\maketitle
\end{verbatim}
プリアンブルで設定した\verb|\title|,\verb|\author|,\verb|\date|の内容を出力する。

\subsubsection{目次の出力}
\begin{verbatim}
\tableofcontents
\end{verbatim}
\verb|\section|,\verb|\subsection|などの見出しから自動的に目次を生成する。
目次を正しく出力するには2回以上のコンパイルが必要。

\subsubsection{改ページ}
\begin{verbatim}
\newpage
\end{verbatim}
強制的に改ページする。目次の後などで使う。

\section{段落レベル・セクション}
見出しレベル。
\subsection{見出し・サブセクション}
サブセクションのレベル
\subsubsection{小見出し・サブサブセクション}
サブサブサブセクションのレベル
\paragraph{小小見出し・パラグラフ}パラグラフはこうなる。
\section{段落レベル・セクション}
見出しレベル2
\subsection{見出し・サブセクション}
サブセクションのレベル2

\section{数式の書き方}
文中に書くときは$y=f(x)$のようにドルマークでくくる。ドルマークが二つになると一行独立する。
$$ y=ax+b $$
ただしこの書き方はあまり推奨されておらず,今は
\[ f=ma \]
のように書くと良い。

数式でよく使う,上付き,下付きはそれぞれ\texttt{\^},\texttt{\_}を使う。

\[ \sum_{i=1}^{n} i=\frac{n(n+1)}{2} \]

ギリシア文字,特殊文字は\texttt{\textbackslash}に続けて名前を書く。$\beta_0$など。\LaTeX もコマンドででる。詳しくはネットでググるといい。

ベクトルや行列は\verb|bm|パッケージを読み込んで,\bm{v}や\bm{A}のように書くと良い。

また、\texttt{equation}環境を使うこともできる。
\begin{equation}
E=mc^2
\end{equation}

\section{リストの書き方}
リストの書き方は3つある。

\subsection{箇条書き(itemize)}
\texttt{itemize}環境を使う。
\begin{itemize}
    \item 最初の項目
    \item 2番目の項目
    \item 3番目の項目
\end{itemize}

\subsection{番号付きリスト(enumerate)}
\texttt{enumerate}環境を使う。
\begin{enumerate}
    \item 手順1:データを収集する
    \item 手順2:データを分析する
    \item 手順3:結果を報告する
\end{enumerate}

\subsection{定義リスト(description)}
\texttt{description}環境を使う。用語の定義などに便利。
\begin{description}
    \item[信頼性] 測定の一貫性・安定性を表す指標
    \item[妥当性] 測定したいものを正しく測定しているかを表す指標
    \item[客観性] 測定者によらず同じ結果が得られるかを表す指標
\end{description}

\subsection{ネストしたリスト}
リストは入れ子にすることができる。
\begin{itemize}
    \item 大項目1
    \begin{itemize}
        \item 中項目1-1
        \item 中項目1-2
        \begin{itemize}
            \item 小項目1-2-1
            \item 小項目1-2-2
        \end{itemize}
    \end{itemize}
    \item 大項目2
\end{itemize}

番号付きリストもネストできる。
\begin{enumerate}
    \item 第1章
    \begin{enumerate}
        \item 第1節
        \item 第2節
        \begin{enumerate}
            \item 第1項
            \item 第2項
        \end{enumerate}
    \end{enumerate}
    \item 第2章
\end{enumerate}

異なる種類のリストを組み合わせることもできる。
\begin{enumerate}
    \item 研究の目的
    \begin{itemize}
        \item 仮説1の検証
        \item 仮説2の検証
    \end{itemize}
    \item 研究の方法
    \begin{description}
        \item[参加者] 大学生100名
        \item[手続き] 質問紙調査
    \end{description}
\end{enumerate}

\section{図表の書き方}\label{sec:figure_table}

\subsection{図の挿入}
図を挿入するには\texttt{graphicx}パッケージを使う(プリアンブルで読み込む)。
\texttt{figure}環境で囲むとキャプションや参照ラベルを付けられる。

\begin{figure}[htbp]
    \caption{図のキャプション例}
    \label{fig:example}
    \centering
    \includegraphics[keepaspectratio,width=0.8\textwidth]{sample_figure.png}
\end{figure}

配置オプション\texttt{[htbp]}の意味:
\begin{description}
    \item[h] here(その場所に配置)
    \item[t] top(ページ上部に配置)
    \item[b] bottom(ページ下部に配置)
    \item[p] page(独立したページに配置)
\end{description}

\verb|\includegraphics|のオプション:
\begin{description}
    \item[width=0.8\textbackslash textwidth] 幅をテキスト幅の80\%に指定
    \item[height=5cm] 高さを5cmに指定
    \item[scale=0.5] 元のサイズの50\%に縮小
    \item[keepaspectratio] 縦横比を維持(widthとheightを同時指定した場合に有効)
    \item[angle=90] 90度回転
    \item[trim=左 下 右 上, clip] 指定した余白を切り取る(単位はbp)
\end{description}

\subsection{表の作成}\label{subsec:table}
表は\texttt{table}環境と\texttt{tabular}環境を組み合わせて作る。

\begin{table}[htbp]
    \caption{記述統計量の例}
    \label{tab:descriptive}
    \centering
    \begin{tabular}{lrrr}
        \hline
        変数 & $M$ & $SD$ & $n$ \\
        \hline
        年齢 & 20.5 & 1.2 & 100 \\
        テスト得点 & 75.3 & 12.4 & 100 \\
        反応時間(ms) & 450.2 & 85.6 & 100 \\
        \hline
    \end{tabular}
\end{table}

\texttt{tabular}の列指定:
\begin{description}
    \item[l] 左揃え
    \item[c] 中央揃え
    \item[r] 右揃え
    \item[|] 縦罫線(APAスタイルでは使わない)
\end{description}

より複雑な表の例:
\begin{table}[htbp]
    \caption{相関行列の例}
    \label{tab:correlation}
    \centering
    \begin{tabular}{lccc}
        \hline
        & 変数1 & 変数2 & 変数3 \\
        \hline
        変数1 & --- & & \\
        変数2 & .45** & --- & \\
        変数3 & .32* & .58** & --- \\
        \hline
    \end{tabular}

    \medskip
    \raggedright
    \small 注)*$p<.05$, **$p<.01$
\end{table}

\section{相互参照}\label{sec:crossref}
\LaTeX では\verb|\label{}|でラベルを付け,\verb|\ref{}|や\verb|\pageref{}|で参照できる。

\subsection{セクションの参照}
このセクションは第\ref{sec:crossref}節である。
図表の書き方は第\ref{sec:figure_table}節(\pageref{sec:figure_table}ページ)を参照。
表の作成については第\ref{subsec:table}項で説明した。

\subsection{図の参照}
図\ref{fig:example}に例を示した。
この図は\pageref{fig:example}ページにある。

\subsection{表の参照}
表\ref{tab:descriptive}に記述統計量を示した。
また,表\ref{tab:correlation}(\pageref{tab:correlation}ページ)には相関行列を示した。

\subsection{数式の参照}
\texttt{equation}環境にもラベルを付けられる。
\begin{equation}
    \bar{x} = \frac{1}{n}\sum_{i=1}^{n}x_i
    \label{eq:mean}
\end{equation}
平均値の定義は式(\ref{eq:mean})の通りである。

\subsection{ラベル命名の慣習}
ラベルには接頭辞を付けると管理しやすい。
\begin{description}
    \item[sec:] セクション(例:\texttt{sec:introduction})
    \item[subsec:] サブセクション(例:\texttt{subsec:method})
    \item[fig:] 図(例:\texttt{fig:result1})
    \item[tab:] 表(例:\texttt{tab:descriptive})
    \item[eq:] 数式(例:\texttt{eq:regression})
\end{description}

\section{引用文献の書き方}

\subsection{概要}
引用文献はBibLaTeX + Biberシステムを使う。BibTeXファイル(拡張子.bib)を用意して,\verb|\addbibresource{ファイル名.bib}|で読み込む。コンパイル手順は以下の通り:
\begin{enumerate}
    \item LuaLaTeXでコンパイル(参照箇所を抽出)
    \item Biberで引用文献リストを作成
    \item LuaLaTeXでコンパイル(引用文献リストを組み込み)
    \item LuaLaTeXでコンパイル(相互参照を解決)
\end{enumerate}

\subsection{JPAスタイルの導入}
JPAに沿った引用文献スタイルは\url{https://github.com/sbtseiji/biblatex-jpa}から入手する。
展開して得られる\texttt{jpa.bbx},\texttt{jpa.cbx},\texttt{jpa.dbx}を同じディレクトリに置く。
プリアンブルで以下のように指定:
\begin{verbatim}
\usepackage[backend=biber,style=jpa]{biblatex}
\addbibresource{ファイル名.bib}
\end{verbatim}

\subsection{Bibファイルの書き方}

各フィールドの基本的な意味:
\begin{description}
    \item[@タイプ] 文献の種類(article, book, inbook, online, softwareなど)
    \item[引用キー] 文献を特定するキー(例:\texttt{Haebara1987})
    \item[author] 著者名。日本語は「姓,名」の形式で
    \item[title] タイトル
    \item[date] 出版年(\texttt{year}より\texttt{date}推奨)
    \item[language] 日本語文献は\texttt{japanese}を指定
    \item[sortname] 日本語著者のソート用ローマ字表記(後述)
\end{description}

ネットからコピペすると不完全なことが多いので、随時補完・修正すること。

\paragraph{sortnameについて}
引用文献リストはアルファベット順に並ぶが,日本語の著者名はそのままではソートできない。
\texttt{sortname}フィールドに著者名のローマ字表記を指定することで,正しい順序でソートされる。

\begin{verbatim}
author   = {南風原,朝和 and 芝,祐順},
sortname = {Haebara,Tomokazu and Shiba, Sukeyori},
\end{verbatim}

注意点:
\begin{itemize}
    \item \texttt{author}と同じ形式(姓,名 and 姓,名)で記述する
    \item ヘボン式ローマ字を使用する
    \item 団体名の場合も同様に指定(例:\texttt{sortname = \{Nihon Shinrigakukai\}})
    \item \texttt{sortname}がないと,日本語文献がリストの末尾にまとめて表示される
\end{itemize}

\subsubsection{英語論文(article)}
\begin{verbatim}
@article{abelson1954technique,
    author  = {Abelson, Robert P},
    title   = {A technique and a model for multi-dimensional
               attitude scaling},
    journal = {Public Opinion Quarterly},
    volume  = {18},
    number  = {4},
    pages   = {405--418},
    date    = {1954}
}
\end{verbatim}

\subsubsection{日本語論文(article)}
日本語文献には\texttt{language = \{japanese\}}と\texttt{sortname}を追加する。
\begin{verbatim}
@article{Haebara1987,
    author   = {南風原,朝和 and 芝,祐順},
    sortname = {Haebara,Tomokazu and Shiba, Sukeyori},
    title    = {相関係数および平均値差の解釈のための確率的な指標},
    journal  = {教育心理学研究},
    volume   = {35},
    number   = {3},
    pages    = {259-265},
    date     = {1987},
    doi      = {10.5926/jjep1953.35.3_259},
    language = {japanese}
}
\end{verbatim}

\subsubsection{英語の本(book)}
\begin{verbatim}
@book{Borsboom2005,
    author    = {Borsboom, Denny},
    title     = {Measuring the mind: conceptual issues in
                 contemporary psychometrics},
    publisher = {Cambridge University Press},
    date      = {2005}
}
\end{verbatim}

\subsubsection{日本語の本(book)}
\begin{verbatim}
@book{anatomia,
    author    = {吉森,護},
    sortname  = {Yoshimori,Mamoru},
    title     = {アナトミア社会心理学},
    subtitle  = {社会心理学のこれまでとこれから},
    publisher = {北大路書房},
    date      = {2002},
    language  = {japanese}
}
\end{verbatim}

\subsubsection{翻訳書}
翻訳書は\texttt{translator},\texttt{origauthor},\texttt{origdate}などを使う。
\begin{verbatim}
@book{Annette_J_Dobson2008-09-08,
    author         = {ドブソン, A.J.},
    sortname       = {Dobson, A.J.},
    origauthor     = {Dobson, Annette Jane},
    origdate       = {2008},
    origtitle      = {Annette Jane Dobsons},
    translator     = {田中, 豊 and 森川, 敏彦 and 山中, 竹春
                      and 冨田, 誠},
    translatortype = {訳},
    title          = {一般化線形モデル入門},
    edition        = {原著第2版},
    origpublisher  = {Chapman \& Hall/CRC Press},
    publisher      = {共立出版},
    date           = {2021},
    language       = {japanese}
}
\end{verbatim}

\subsubsection{本の一章(inbook)}
\begin{verbatim}
@inbook{Allport1935,
    author    = {Allport, Gordon William},
    title     = {Attitudes},
    booktitle = {Readings in Attitude Theory and Measurement},
    editor    = {Martin Fishbein},
    publisher = {John Wiley \& Sons Inc},
    address   = {New York},
    pages     = {3--13},
    date      = {1967}
}
\end{verbatim}

\subsubsection{オンライン資料(online)}
\texttt{url}と\texttt{urldate}(アクセス日)を必ず記載する。
\begin{verbatim}
@online{jpa_manual2022,
    author   = {{日本心理学会}},
    sortname = {Nihon Shinrigakukai},
    title    = {執筆・投稿の手びき(2022年版)},
    url      = {https://psych.or.jp/manual/},
    urldate  = {2025-01-07},
    date     = {2022},
    language = {japanese}
}
\end{verbatim}
団体名は二重波括弧\verb|{{...}}|で囲むと姓名分割されない。

\subsubsection{ソフトウェア(software)}
\begin{verbatim}
@software{R2024,
    author  = {{R Core Team}},
    title   = {R: A Language and Environment for
               Statistical Computing},
    url     = {https://www.R-project.org/},
    version = {4.4.0},
    date    = {2024},
    address = {Vienna, Austria}
}
\end{verbatim}

\subsection{引用コマンド}
\begin{description}
    \item[\textbackslash textcite\{\}] 本文中に著者名を出す:\textcite{abelson1954technique}
    \item[\textbackslash parencite\{\}] 括弧内に著者名と年:\parencite{Borsboom2005}
    \item[複数文献] カンマ区切りで指定:\textcite{Haebara1987,anatomia}
\end{description}

オンライン資料やソフトウェアも同様に引用できる:\parencite{jpa_manual2022,R2024,jamoviJP}

\subsection{引用文献リストの出力}
文書の最後に次のコマンドを書く:
\begin{verbatim}
\printbibliography[title=引用文献]
\end{verbatim}

\printbibliography[title=引用文献]

\end{document}